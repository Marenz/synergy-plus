\section{Methodology}

Many developers before now have vowed to rewrite Synergy. Unfortunately though,
the intention has never materialized due to lack of motivation, involvement from
the community, and momentum.
To solve this problem, we should plan the new version thoroughly and use the
Scrum Agile methodology which will allow us to deliver features in small 
bursts (which will maintain momentum and community interest).

In version 1, we use Buildbot for the nightly build, so this can be used to run
the integration tests (as explained later in this section) as well as standard 
unit tests (which are also run by developers).

\includegraphics{uml/meth-testing.1}

\subsection{Unit Testing}

Open source software should be easy for new developers to modify without too
much risk of breaking the existing functionality. Currently, applying patches 
is very risky, since this usually causes regressions. Unit testing will help 
developers identify regressions before we check code in.

However, some aspects of Synergy are difficult to test, since it always requires
more than one machine to function. We \textit{could} emulate this behavior on a 
single machine by running both the server and client (so that they communicate 
locally), but this only allows us to test the network layer (not the mouse,
keyboard, clipboard, and so on). The solution is to use integration testing
\ldots

\subsection{Integration Testing}

Actually having to fix regressions is not usually a problem -- the problem is
finding them in the bug in first place, which often falls on the end user. 
Often, the regression will be caused way outside of the developer's capable 
testing scope (e.g. on another platform). The solution is to delegate cross-
platform testing to an automated process.

Nightly integration testing (in the form of unit tests) will be run on on the 
Buildbot build machines, and will
simulate a fully fledged environment involving many \textit{real} machines. The 
contributor would never run these tests before committing (it wouldn't be 
practical), but it will help detect problems early enough for us to establish
blame on a particular source code change.

In each test environment, there will be 3 Buildbot slaves (1 server, 2 clients).
To run a test, the Buildbot slaves must make some assumptions; that the clients
can communicate with the server via LAN, and that they are all turned on at the 
same time (we will have to do this via a Buildbot schedule). Since we'd be 
making timing assumptions, there is a large margin for error, so the test node
sync operation must produce helpful errors (and check the environment fully).

All integration test cases must be implemented \textbf{before} we start 
implementing any functionality (otherwise they might never get round to writing
them). Because the integration testing is so important, it is effectively our 
Achilles Heel. We must understand that the success of the project (and future 
maintenance) depends very heavily on the build system.

Each integration unit test should do the following:

\begin{enumerate}
  \item Synchronize with all other nodes.
  \item Start either a client or server.
  \item Either wait for client connections, or connect to the server.
  \item Have the server test node move the mouse ``off the screen''.
  \item Pass or fail depending on whether or not client mouse pointer has
    actually moved to the expected coordinates.
\end{enumerate}

We must design this test each time for testing the clipboard, keyboard input,
screensaver sync, and all other major features.

Each test should run once for every possible combination of platform. For 
example:

\begin{itemize}
  \item Windows server, Linux client, Mac OS X client.
  \item Windows server, 2 Windows clients.
  \item Linux server, Windows client, Mac OS X client.
  \item \ldots
\end{itemize}

\subsection{User Stores}

The problem with defining stories for multi-platform, client-server software
is that most features must be implemented once for each platform, then for both
client and server (even more so when the software is integrated tightly with
each platform). This means we need to implement many stories up to 6 times.

\textit{Note: This is not an exhaustive list of features.}

\textbf{As a user:}

\begin{itemize}
  \item I want to move my server mouse to a client computer (and back) so that 
    I only need one mouse.
  \item I want keys pressed on the server to go to the same computer as the
    mouse cursor so that I only need one keyboard.
  \item I want data copied to my clipboard to replicate to the computer which
    I move my cursor to, so that I can copy data between computers.
  \item I want screen savers on all connected computers to start at the same
    time so that all screens stay on while I'm working.
  \item \ldots
\end{itemize}

\textbf{For each:}

\begin{itemize}
  \item Windows server
  \item Windows client
  \item Linux server
  \item Linux client
  \item Mac OS X server
  \item Mac OS X client
\end{itemize}

With these very simple looking user stories, we should be able to make a good 
start, but even the first release must work (if only with little platform
support).

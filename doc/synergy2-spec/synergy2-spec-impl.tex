\section{Implementation}

\subsection{Language}

Our language of choice will be C++ as with version 1. However, we might want to
consider writing the GUI in Python (Chris's idea). Regardless, we should 
separate the core system layer and the GUI layer via IPC, so that we can use
any language to write the GUI.

\subsection{Libraries}

We should use Boost to increase productivity; typically an application would
take longer to write in C++ because the standard library has less functionality
available to the developer - so we spend more time reinventing-the-wheel.

\subsection{Toolchain}

Historically, Automake was used before Synergy+ came along and started using
CMake (each have pros and cons). Because CMake is a little limited, a hm.py 
script was created to extend the toolchain so that less input from the developer
was required.

\textbf{TODO:} Should we stick with the Synergy+ way?

\subsection{Code Style}

We'll use the code style from version 1, with a few modifications. For example 
we will drop the C-prefix class naming convention (adopted from MFC). See 
appendix A for the full guide.

\subsection{Debugging}

Debugging Synergy is quite difficult, since doing so requires the use of the 
same mouse and keyboard that Synergy is using. Let's say we break in the 
function that handles cursor movements, often doing so will cause the mouse
to freeze for a while until the hook is released automatically.

We should design version 2 so that it is easy to debug and test features. It
might be that the existing code runs the HID input handling in a separate 
thread -- in this case, we need to make it obvious in our documentation not to 
break all threads (and how to do this in various debuggers).

\subsection{Logging}

Logging is very important for post-mortem analysis, but is also useful during
development since it can be so difficult to debug. File logging should be 
enabled by default for the end-user, with only warnings and errors, with some
form of log  rotation to conserve user disk space.

\section{Code Style}

\subsection{New in version 2}

If you're already familiar with the version 1 syntax, then here's all you need
to know:

\begin{enumerate}
  \item No C prefix on classes.
  \item 2 space indentation, not tabs.
  \item Put function modifiers on the same line.
\end{enumerate}

\subsection{Style rules}

\subsubsection{Space indentation}

\begin{verbatim}
\s\s printf("hello world"); // correct
\t printf("hello world"); // incorrect
\end{verbatim}

\subsubsection{Class, struct, and enum names are upper camel case}

\begin{verbatim}
class MyClass { };
struct MyStruct { };
enum MyEnum { };
\end{verbatim}

\subsubsection{Member, and enum constant names are lower camel case}

\begin{verbatim}
class MyClass {
public:
    void helloWorld();
};
\end{verbatim}

\subsubsection{Member variable prefix}

\begin{verbatim}
class MyClass {
public:
    MyClass* m_myClass1;
private:
    MyClass* m_myClass2;
};
\end{verbatim}

\subsubsection{Enum constants are prefixed with k}

\begin{verbatim}
enum MyEnum {
    Value1,
    Value2
};
\end{verbatim}

\subsubsection{Static variable prefix}

\begin{verbatim}
static MyClass* s_myClass;
\end{verbatim}

\subsubsection{Comments and debug messages need not be grammatically correct}

\begin{verbatim}
// i'm using bad grammar. but i like to use full stops
DEBUG((CLOG_INFO "hello world"));
\end{verbatim}

\subsubsection{Multi-line comments use single-line commenting (//)}

\begin{verbatim}
// for long block comments, instead of using the slash asterisk
// comments, we use the two-slash comments
\end{verbatim}

\subsubsection{Function return types must go on the line above the function name}

\begin{verbatim}
// correct
int MyClass::helloWorld() {
  // ...
}

// incorrect
int
MyClass::helloWorld() {
  // ...
}
\end{verbatim}

\subsubsection{Curly braces start on the same line as the function}

\begin{verbatim}
// correct
int MyClass::helloWorld() {
  // ...
}

// incorrect
int MyClass::helloWorld()
{
  // ...
}
\end{verbatim}

\subsubsection{Curly braces start on the same line for statements}

\begin{verbatim}
// correct
if (a == b) {
    // ...
}

// correct
for (int i = 0; i < a; i++) {
    // ...
}

// incorrect
if (a == b)
{
    // ...
}

// incorrect
for (int i = 0; i < a; i++)
{
    // ...
}
\end{verbatim}

\subsubsection{A space is used between operators and operands}

\begin{verbatim}
// correct
a == b;
if (c == d) {
  // ...
}

// incorrect
a=b;
if (c==d) {
  // ...
}
\end{verbatim}

\subsubsection{No space is inserted after or before conditions within parenthesis}

\begin{verbatim}
// correct
if (a == b) {
    // ...
}

// incorrect
if ( a == b ) {
    // ...
}
\end{verbatim}

\subsubsection{Don't use void keyword for functions with no parameters}

\begin{verbatim}
// correct
void helloWorld();

// incorrect
void helloWorld(void);
\end{verbatim}

\subsubsection{The const keyword is used regularly}

\begin{verbatim}
class MyClass {
public:
    const char* helloWorld();
};

MyClass::MyClass(const char* helloWorld)
{
  // ...
}
\end{verbatim}

\subsubsection{Space between statement parenthesis and the keyword}

\begin{verbatim}
// correct
if (a == b) {
  // ...
}

// incorrect
if(a == b) {
  // ...
}
\end{verbatim}

\subsubsection{No spaces between type and reference and pointer operator}

\begin{verbatim}
// correct
char* helloWorld;

// incorrect
char *helloWorld;
\end{verbatim}

\subsubsection{Pre-processor command indentation}

\begin{verbatim}
// correct
#if HELLO_WORLD
#include "HellWorld.h"
#endif

// incorrect
#if HELLO_WORLD
#  include "HellWorld.h"
#endif
\end{verbatim}

\subsubsection{Pointers and references do not have prefixes}

\begin{verbatim}
// correct
MyClass* myClass1 = new MyClass();
MyClass& myClass2 = *myClass1;

// incorrect
MyClass* p_myClass1 = new MyClass();
MyClass& p_myClass2 = *p_myClass1;

// incorrect
MyClass* pMyClass1 = new MyClass();
MyClass& pMyClass2 = *pMyClass1;
\end{verbatim}

\subsubsection{The left comparator operand is a variable}

\begin{verbatim}
// correct
if (a == "hello world") {
  // ...
}

// incorrect
if ("hello world" == a) {
  // ...
}
\end{verbatim}

\subsubsection{The else statement on the same line as the curly brace}

\begin{verbatim}
// correct
if (a == b) {
  // ...
} else {
  // ...
}

// incorrect
if (a == b) {
  // ...
}
else {
  // ...
}
\end{verbatim}

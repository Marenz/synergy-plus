\section{Design}

\subsection{Scope}

With a tool as conceptually simple as Synergy, it's very easy to get carried
away with feature requests. Some features we will \textbf{not include}:

\begin{enumerate}
  \item Dragging windows between screens.
  \item Remote audio sharing.
  \item Connecting over the Internet.
  \item Smart phone or tablet as server.
  \item Remote desktop integration.
\end{enumerate}

\subsection{GUI}

While version 1 is simple to setup, it can be a little difficult to
troubleshoot. As command-line tools go, the existing version is great, but
can be a little involved for some users. Often these users are from a Windows
or Mac background where tool configuration is usually done with a GUI.

Version 1 was originally designed as a command line tool, then later on, several
GUI applications were invented to try and make configuration easier. QSynergy is
one of those GUIs that found a good balance of cross-platform support and user 
friendliness. However, like all Synergy GUIs, it is limited by the lack of 
interoperability with the underlying command line utility. So we should design 
version 2 with inter-process communication (IPC) to the GUI firmly in mind. 
Security is also an issue; we need to make sure a malicious program can't
connect to Synergy via the IPC layer (perhaps a shared key will be sufficient).

With this version, we aim to make configuration extremely simple so that anyone
can use Synergy without having to know anything about IPs or which machine is a
server or a client. One setup option will be that the user can use a wizard (see
Figure \ref{fig:setupWizard}). The wizard should include an ``online'' manual 
with a troubleshooting section. Troubleshooting steps might include asking the 
user to temporarily turn off their firewall, try pinging, etc. This will reduce
the number of troubleshooting requests coming into the website and mailing list.
The user will can always drag and drop screens into the designer (instead of
using the edge-hit feature) just in case the edge-hit feature fails to work
correctly.

When starting the wizard, the user will choose between two options: Simple (a
guided process, which is very easy to use, and creates a P2P network); and
Advanced, which allows the user to load a config file -- this makes it possible
to setup a sever-client configuration, or lead a version 1 config.

\begin{figure}[ht!]
  \centering
  \includegraphics{diag/ad-wizard.1}
  \caption{Setup wizard}
  \label{fig:setupWizard}
\end{figure}

\subsubsection{Place screens anywhere}

While the version 1 configuration syntax does allow for screens to be placed at
positions relative to each other, most (if not all) GUIs so far have not 
considered this. In version 2 we will allow the user to drag screens anywhere
(on the layout designer) so that you can have (for example) two small screens
stacked next to a large screen.

\subsection{Run at startup}

With version 1, most users need to have some technical knowledge of their OS
in order to start synergy when their computer boots. This new version must
take care of starting at boot automatically in order to be more seamless.

Autostart on Linux is exceptionally difficult because different versions Linux 
require different configurations. In particular auto starting after the X server
starts requires editing shell scripts. Maybe we can start via init.d and poll 
for a running X server.

\subsection{Keyboard}

In version 1, key chars are sent over the network. This poses a nightmare when
it comes to multilingual support, or even just supporting new keys (e.g. media
keys, etc). It's easy to read key chars, and it's tempting to re-use this 
approach in version 2; however, it will only give us problems and bugs to fix,
when what we really need to do is support key codes.

What's so good about key codes? Instead of relaying the char that the user would
see typed in any other application, we relay the code of the key that is 
pressed, which is not be tied to any particular language. This means that a user
can type in English on computer A, and Russian on computer B without needing to
switch the language on the master. Instead, the user chooses the language on
the receiver.

There is a complication with key codes however; each operating system has its
own set of key codes - so we need to map each key code to our own set, so that
key codes sent over the network are generic and can be translated to any 
platform.

\subsection{Network model}

We will support both server-client (traditional) and a new P2P model, as
described in this section.

\subsubsection{Server-client}

Historically, Synergy has always used a server-client model, where the server's
mouse and keyboard control the clients (figure \ref{fig:serverClient}). This is
actually the opposite to how remote access applications work, where the computer
being controlled is the server (not a client). We will use the same port as 
version 1 (24800), though the protocols will not be compatible.

\begin{figure}[ht!]
  \centering
  \input diag/nd-serverclient.tex
  \caption{Server-client network model}
  \label{fig:serverClient}
\end{figure}

The computer with the mouse and keyboard must be online at all other times to
make the other computers useful (laptops excluded). To some degree, it makes 
sense to have this computer as the server, since the server should be online
at all times.

The disadvantage is that when the server computer goes offline, all other 
computers become somewhat redundant (until you physically plug a keyboard
and mouse into each of them). One solution is to allow any computer to be a 
master (to have control over the other computers)...

\subsubsection{Peer-to-peer (P2P)}

% this section is heavily based on this mailing list thread:
%   http://groups.google.com/group/synergy-plus-dev/browse_thread/thread/5a09353d55fc1364

Allowing any computer to be a master implies that any computer can control
any other computer. So you could have two computers with a mouse (e.g. a 
desktop and a laptop) and two without. Figure \ref{fig:p2p} shows 3 networks;
each show 3 nodes listening for TCP connections, and one (the hub) making the
outgoing connections. This figure also illustrates that any node can be either
a master, or a hub, or both (the terms ``hub'' and ``master'' are not to be 
confused with each other).

\begin{figure}[ht!]
  \centering
  \input diag/nd-p2p.tex
  \caption{P2P ``hub'' network model}
  \label{fig:p2p}
\end{figure}

Why might this be useful? Let's say you've just arrived at the
office, with your laptop resumed from sleep - you sit down just in time to see
a 30 second restart countdown on your ``mouseless'' computer. Uh-oh, looks like
that new sys admin has decided to restart your workstation, great! No problem, 
just fire up your Synergy server machine, move the mouse over and hit the 
``Abort'' button - but oh no, your server machine is running Windows and has 
decided to blue screen today. Your 2nd computer restarts and you've lost an 
entire night's work because you forgot to hit save at 3am. Ok, so it's a bit 
of an edge-case but you get the picture; some times it's a chore to turn your
server on in order to use your other computers.

We might be tempted to have each node connect to every other node, but
unfortunately this doesn't scale too well; it's always better to have less
connections if it can be helped. The hub topology is well suited for this,
since it will make the smallest number of connections.

This topology will work well on DHCP networks where
each computer could potentially have a new IP each time it reconnects (though
the rate at which IPs change is quite slow/predictable). Even so, this also 
means that the user need not have a valid DNS setup (which is a fairly common
problem). Many first-time users attempt to use only the hostname (which actually
resolves to nothing at all), then complain when they cannot connect. The answer
is always ``Use IP addresses instead,'' which needless to say, is very tiring.

\subsubsection{The master mode}

So when does a node switch to master mode? One option might be to make this
automatic and triggered by mouse movement. The problem with this sort of
detection is that it's not always possible (depending on the platform) to
detect real (by a human) or ``fake'' (by Synergy) input. It might be the
case that some OSs can distinguish between real and fake input, but we cannot
always guarantee this to be the case. So, we should make master/slave switching
manual at the user's discretion. For example, if the user wishes for their
laptop to become the master, they can just use their laptop touch pad to
select master from their system tray icon (or command line, etc). It is
important that the user can do this while Synergy is running; having to restart
the program may make things complicated for the user. We may decide to 
introduce an automatic master mode at some point in the future, but this is
certainly not essential for the meantime.

\subsubsection{Introducing new nodes}

% nb: hmm, chris's idea seems to replace this
%Each time a new node joins the configuration, the hub node will connect to 
%it. If the node announces but does not get any incoming connections from an
%existing hub node, then it will assume the hub role. A non-hub will always send
%messages through the hub node (which is a TCP client), since all other nodes 
%are always connected via the hub. All nodes will attempt to configure 
%themselves automatically without requiring any intervention from the user. This
%does require some degree of guesswork; we will need to make the program 
%``smart'' enough to figure out which node to use as the hub, and so on. This of
%course introduces a very large margin for error. So as a backup, we should 
%allow the user to manually force a particular node to be the hub, in case the
%automatic configuration does not work correctly.

% chris has seen a flaw in this design:
%  ``if two nodes, A and B, each announce it's the new hub then a third node, C, 
%  might see A's announcement before B's but a fourth node, D, might see B's first.
%  the algorithm for choosing a hub must be robust against this.  this is where 
% research into other P2P software might yield some quick solutions.
%If a non-hub node looses connection with the hub, it will send an announce (just
%like it did when it first came online). This may have happened for one of two
%%reasons: either the hub has gone offline; or the connection between the node
%and the hub has been interrupted (maybe the node's firewall was reconfigured).
%For the former case (hub went offline), all nodes will be affected by this, so
%eventually one of them will announce (just like when it first came online).
%Since none of the nodes are hubs, none can respond, therefore the first node
%to disconnect becomes the new hub, and will respond to further announces.
%For the latter case (connection interrupted), the node will still announce,
%but it's likely that even if the hub is online, it probably can't communicate
%with the non-hub node. In this case, the orphaned node will make a special
%kind of announcement, to say ``I'm the new hub!'' -- in this case, all other
%nodes (both hub and non-hub) will drop their connections and announce. The
%new hub will then pick up all other nodes. If none respond, then we can assume
%that either there is a serious network error, or there genuinely are no more
%nodes. In the event that two nodes attempt to become the hub at nearly the same
%time, there will always be one that does so a little later than the other; in
%this case, the latest node will become the hub (we may want to introduce a time
%delay based on node ID to prevent this from happening).

% TODO: diagram would make this easier to visualise
% ... instead, chris has proposed this approach (seems to be the opposite):
What if a node looses connection? One approach is for every node to claim to 
be the hub when it starts (or gets disconnected). Nodes with an already 
established connection to a hub will ignore the claim. Any existing hub
(including newly started nodes) will evaluate the claim. If the claimant node,
H, has, say, an IP or MAC address that's smaller than that of local node, L,
then L will authenticate with H. If that succeeds then L will disconnect from 
all other nodes and use H as its hub. The newly disconnected nodes will then 
look for a new hub; they may get another claim from H, remember recent claims, 
or L can tell them about H before disconnecting. Note that L must authenticate 
with H to prevent a DoS, otherwise any node on the network could claim to have 
the lowest IP/MAC address and discard messages.

% nb: chris says node ids don't need to be sequential. so do they need to be random?
%The hub node will be aware of all other non-hub nodes by IP address, and will
%associate each one with a sequential 16-bit node ID. The non-hub nodes themselves
%do not need to know the IP addresses of other non-hub nodes, since the hub node
%will route messages based on node ID. The full set of node IDs are maintained
%on every node, and are pushed out from the hub each time a new node leaves or
%joins.

Node IDs are held only on the hub node. When a new hub comes online, it will
start it's own new list of IDs and IP addresses.

% nb: seems to be incorrect now
%If a node loses contact with its peer, it will attempt to automatically fix
%the problem in one of two ways (depending on whether the incoming or outgoing 
%connection was lost). If the outgoing connection was dropped (i.e. the 
%connection that the node has initiated) then it will attempt to connect each 
%of the other known nodes in round-robin fashion until it finds a new peer. It 
%will keep trying forever, or until the incoming connection drops (if this 
%happens then we can assume that the IP-to-node mapping may be stale). If
%the incoming connection drops, then the node will re-announce (and will keep
%doing so forever), hoping that another Synergy node will eventually connect.

\subsubsection{Security concerns}

Allowing any Synergy node to say ``Hey, I'm here, connect to me!'' introduces
a security dilemma. A malicious program could easily broadcast using Bonjour
pretending to be a real Synergy client. We should do two things to help prevent
this: first, have a configurable option to turn remote control on or off for
a particular node (we might turn this off by default, and allow users to turn
it on during configuration); secondly, messages can be encrypted (on by 
default).

\subsubsection{Legacy support}

The disadvantage of P2P is that some users may have their server on one network
(behind a firewall) and a client on a restricted network (e.g. corporate wifi).
For this reason, we must still support client-server for such users who are 
less fortunate. Another reason to still support client-server is that users may
want to upgrade straight from version 1 to version 2 and use the same config 
file.

\subsection{Protocol}

\subsubsection{TCP multiplexing}

Synergy relies on user input being relayed in real time; the mouse and keyboard
movements must be responsive. Version 1 uses a single TCP channel for both user
input and clipboard data. This can be a problem when a large amount of clipboard
data is copied to a client, as this can cause the cursor to stall until the
clipboard data has been copied.

A solution is to implement TCP multiplexing (mux), which will 
allow us to prioritize the messages that we send over a single connection. We
will achieve this by chunking messages and assigning them to a prioritized channel,
where human input device (HID) messages are the highest priority. The
receiver will then be responsible for reconstructing the chunks into full 
messages. However, we might want to be careful of chunking HID messages, since 
these need to be more performant (the speed of sending clipboard data doesn't 
really matter).

On the receiver, each mux channel has a buffer for incomming chunks, a single
variable for the current incomming message header. As the chunks arrive, they
are appended to the correct channel buffer. We will assume that chunks always
arrive in the correct order, and given this, we can easily reconstruct full
messages when all chunks are received. We will take a full message when the
chunk buffer reaches the correct message size (as indicated by the current
message header). If a new header is received before a full message is taken
from the chunk buffer, then the current buffer is discarded (this should be
logged).

\subsubsection{Message format}

There are two methods of transporting data; \textit{lightweight} (see Table
\ref{tab:lightMessage}) and \textit{multiplexed} (see Table
\ref{tab:muxHeadMessage} and \ref{tab:muxChunkMessage}). We should not use 
chunks for smaller messages, since chunking has some overhead. The lightweight
message will be used for real time data (e.g. mouse movement deltas), and the
mux message will be used for any other data (e.g. clipboard).

We need to identify whether a message is lightweight or mux. Since all
messages are small we only need 7 bits to indicate size (up to 127 bytes).
If the first bit of the message (the high bit of the size) is 0xff then 
it's a lightweight message. You can't express message sizes $\geq$ 0xff000000.

The total mux buffer (to which all chunks are appended) can be 4GB to allow 
for large data. This will allow us to implement copy/paste for large files
(as a quick alternative to using file shares), however we must limit this
to a configurable limit (default of 1GB) so that memory use is sensible.

\begin{table}[ht!]
  \begin{tabular}{|l|l|l|l|l|}
    \hline
    \textbf{Pos (B)} &
    \textbf{Bit field} &
    \textbf{Part} &
    \textbf{Size} &
    \textbf{Comment} \\
    \hline
    0 & 7-2 & Size & 7b/1B & Size inc header (max $2^7-1$ bytes), \\
      &     &      &       & and type (high byte == 0xff). \\
      & 1-0 & Mux ch. & 1b/1B & Lightweight mux channel (0 or 1). \\
    1 & 23-12 & To ID & 12b/3B & Runtime generated recipient node ID. \\
      & 11-0 & From ID & 12b/3B & Runtime generated sender node ID. \\
    4 & & Payload & 123B & Arbitrary data. \\
    \hline
  \end{tabular}
  \caption{Lightweight message format}
  \label{tab:lightMessage}
\end{table}

\begin{table}[ht!]
  \begin{tabular}{|l|l|l|l|l|}
    \hline
    \textbf{Pos (B)} &
    \textbf{Part} &
    \textbf{Size} &
    \textbf{Comment} \\
    \hline
    0 & Size & 4B & Final size of chunk buffer (max $2^{32}-1$ bytes), \\ 
      &      &    & and type (high byte != 0xff). \\
    4 & Mux ch. & 1B & Mux channel (0 to 254). \\
    5 & To ID & 6b/3B & Runtime generated recipient node ID. \\
      & From ID & 6b/3B & Runtime generated sender node ID. \\
    \hline
  \end{tabular}
  \caption{Multiplex head message format}
  \label{tab:muxHeadMessage}
\end{table}

\begin{table}[ht!]
  \begin{tabular}{|l|l|l|l|l|}
    \hline
    \textbf{Pos (B)} &
    \textbf{Part} &
    \textbf{Size} &
    \textbf{Comment} \\
    \hline
    0 & Size & 2B & Chunk size inc header (max $2^{16}-1$ bytes). \\
    2 & Mux ch. & 1B & Mux channel (0 to 254). \\
    3 & Payload & 63KB & Chunk of arbitrary data. \\
    \hline
  \end{tabular}
  \caption{Multiplex chunk message format}
  \label{tab:muxChunkMessage}
\end{table}

\subsubsection{Encryption}

Encryption will be supported, and will be enabled by default. Cryptography may 
put a noticeable load on the CPU, so we must allow this to be turned on and off
(but on by default). The encryption layer is separate to the message protocol.

Security requires not just encryption but also authentication. We should choose
or suggest appropriate standards for these. Do we allow various encryption and
authentication schemes? If so we must define a handshake to negotiate them.

\footnotetext{Human input device}

\subsection{Configuration}

In server-client mode, the configuration will live on the server, just like in
version 1. We will also entirely support version 1 configurations. In fact, 
version 2 configuration file will be almost identical, but with a few added
options (e.g. to state whether or not the setup is client-server or P2P). We 
will not make version 2 configurations backward-compatible.

In P2P mode however, things get complicated. Each node must be aware of its
neighbour in terms of physical screen location and network peering
(incidentally, there will be no correlation between the two, since network 
peering will be based on availability and network time proximity). So, in the
case of P2P, the configuration file will be shared. Since there can only be
one master active at a time, this machine is responsible for distributing
changes to the configuration file. Now that the configuration file is shared,
any machine can assume command of the ship, since it'll already know the layout
of the screens.

\subsubsection{Screen stacking}

Using version 1, if the user adds a node to a Synergy configuration which has
multiple operating systems install (sometimes known as dual-boot if there are
two operating systems), then they have a problem, since you cannot have two
screens directly adjacent to a particular screen (TODO: prove this!). A quick
fix for this would be to give the machine the same hostname or IP, but this
might not always be convenient for the user (depending on their network setup).

One solution is to allow users to add nodes to a stack, which can be placed
adjacent to another screen or stack. This way we can allow one of the screens
in a stack to be active at any given time.

\clearpage
\subsection{UML}

\begin{figure}[ht!]
  \begin{center}
    \includegraphics{diag/cd-system.1}
    \caption{System class diagram, for server and client}
    \label{fig:systemClassDiagram}
  \end{center}
\end{figure}

\begin{figure}[ht!]
  \begin{center}
    \includegraphics{diag/cd-server.1}
    \caption{Server class diagram}
    \label{fig:serverClassDiagram}
  \end{center}
\end{figure}

\begin{figure}[ht!]
  \begin{center}
    \includegraphics{diag/cd-client.1}
    \caption{Client class diagram}
    \label{fig:clientClassDiagram}
  \end{center}
\end{figure}

\begin{table}[ht!]
  \begin{tabular}{|l|l|}
    \hline
    \textbf{Class} & \textbf{Purpose} \\
    \hline
    System & Common functionality between Client and Server. \\
    Server & Contains server-side code. \\
    Client & Contains client-side code. \\
    Screen & The entire viewable area, which could span many monitors. \\
    VirtualScreen & Represents what is displayed on each monitor. \\
    Device & A human input device (e.g. Mouse, Keyboard, Joystick). \\
    Mouse & Captures mouse input, representing the current state. \\
    Keyboard & Captures keyboard input, representing key states. \\
    DeviceRelay & Relays received human input from server to the OS. \\
    MouseRelay & Relays mouse movement to the OS. \\
    KeyboardRelay & Relays pressed keys to the OS. \\
    \hline
  \end{tabular}
  \caption{Class purpose definition}
  \label{tab:classes}
\end{table}

\begin{figure}
  \centering
  \input diag/sd-system.tex
  \caption{Server and client sequence diagram}
\end{figure}

\begin{figure}
  \centering
  \input diag/sd-clipboard.tex
  \caption{Clipboard sharing sequence diagram}
\end{figure}

\section{Design}

\subsection{Scope}

With a tool as conceptually simple as Synergy, it's very easy to get carried
away with feature requests. Some features we will \textbf{not include}:

\begin{enumerate}
  \item Dragging windows between screens.
  \item Remote audio sharing.
  \item Connecting over the Internet.
  \item Smart phone or tablet as server.
  \item Remote desktop integration.
\end{enumerate}

\subsection{GUI}

While version 1 is simple to setup, it can be a little difficult to
troubleshoot. As command-line tools go, the existing version is great, but
can be a little involved for some users. Often these users are from a Windows
or Mac background where tool configuration is usually done with an intuitive
GUI.

Version 1 was originally designed as a command line tool, then later on, several
GUI applications were invented to try and make configuration easier. QSynergy is
one of those GUIs that found a good balance of cross-platform support and user 
friendliness. However, like all Synergy GUIs, it is limited by the lack of 
interoperability with the underlying command line utility. So we should design 
version 2 with inter-process communication (IPC) to the GUI firmly in mind. 
Security is also an issue.

\subsection{Wizard}

While we should support the existing config file format, some would prefer a
setup wizard to simplify the process. The wizard should display immediately
after installation on all platforms (with the option to skip the wizard). When
the wizard starts, the user should choose whether the machine is a client or a
server (the wizard should explain the difference with a simple diagram).

\textbf{TODO:} Replace steps with an activity/sequence diagram.

\subsubsection{Server steps}
\begin{enumerate}
  \item Show instructions for starting client wizard(s).
  \item Show connected clients in a status list (as they connect).
  \item Allow user to advance once all clients have connected.
  \item Allow user to drag and drop clients on a grid.
  \item Save config to file (at custom location) when complete.
\end{enumerate}

Often, the user's hostname will not be found on their DNS server, and in such
a case we should detect this, and ask the user to use an IP instead.

\subsubsection{Client steps}
\begin{enumerate}
  \item Prompt the user for the server hostname or IP (shown in server wizard).
  \item If unable to connect, show troubleshooting tips.
\end{enumerate}

Troubleshooting steps might include asking the user to temporarily turn off
their firewall, try pinging, etc. This will reduce the number of troubleshooting
requests coming into the website and mailing list.

\subsection{Keyboard}

We should support both key chars (same as version 1) and key codes (new and
experimental). If it turns out that there are unforeseen problems with the
new experimental approach, then we can release with the traditional approach.

The advantage of using key codes instead of chars is that each computer manages
it's own internationalization - which is best practice, since most operating
systems manage language per application. Using key codes, you could for example 
use one computer for English and the other for Russian.

There is a complication with key codes however; each operating system has it's
own set of key codes - so we need to map each key code to our own set, so that
key codes sent over the network are generic and can be translated to any 
platform.

\subsection{Protocol}

\textbf{TODO:} Support encryption from the start?

\textbf{TODO:} Chris says to only support key codes, not both chars and codes!

\textbf{TODO:} Talk about TCP multiplexing (for priority; where clipboard data 
is lowest priority, and mouse is highest).

Each key press and mouse delta will trigger a message sent over a simple TCP 
stream (see Table \ref{tab:messageFormat}).

\begin{table}
  \begin{tabular}{|l|l|l|l|l|}
    \hline
    \textbf{Pos} &
    \textbf{Part} &
    \textbf{Size} &
    \textbf{Type} &
    \textbf{Comment} \\
    \hline
    0 & Size & 4B & uint32\_t & The total message size in bytes. \\
    4 & Type & 2B & uint16\_t & See Table \ref{tab:messageTypes} for enum. \\
    6 & Data & $\leq$4GB & char[] & An array of arbitrary binary data. \\
    \hline
  \end{tabular}
  \caption{TCP message format}
  \label{tab:messageFormat}
\end{table}

\begin{table}
  \begin{tabular}{|l|l|l|}
    \hline
    \textbf{Type} &
    \textbf{Data} &
    \textbf{Detail} \\
    \hline
    HelloServer & Client key & Client message to server on connect. \\
    HelloClient & Access allowed? & Response to key authentication. \\
    KeyCodeDown & Key code & Generic HID key code (from keyboard). \\
    KeyCharDown & Unicode char & Actual character the user would see. \\
    \hline
  \end{tabular}
  \caption{Message type enumeration (incomplete)}
  \label{tab:messageTypes}
\end{table}

As discussed in the keyboard section, the protocol layer should support both
chars and key codes, but the message should make a clear distinction for the
mode used.

The most discussed aspect of the protocol is keyboard input; more specifically
whether to send chars or key codes over the network. The reality is, that as far
as the protocol goes, it should easily support either.

\clearpage
\subsection{UML}

\begin{figure}[ht!]
  \begin{center}
    \includegraphics{uml/cd-system.1}
    \caption{System class diagram, for server and client}
    \label{fig:systemClassDiagram}
  \end{center}
\end{figure}

\begin{figure}[ht!]
  \begin{center}
    \includegraphics{uml/cd-server.1}
    \caption{Server class diagram}
    \label{fig:serverClassDiagram}
  \end{center}
\end{figure}

\begin{figure}[ht!]
  \begin{center}
    \includegraphics{uml/cd-client.1}
    \caption{Client class diagram}
    \label{fig:clientClassDiagram}
  \end{center}
\end{figure}

\begin{table}[ht!]
  \begin{tabular}{|l|l|}
    \hline
    \textbf{Class} & \textbf{Purpose} \\
    \hline
    System & Common functionality between Client and Server. \\
    Server & Contains server-side code. \\
    Client & Contains client-side code. \\
    Screen & The entire viewable area, which could span many monitors. \\
    VirtualScreen & Represents what is displayed on each monitor. \\
    Device & A human input device (e.g. Mouse, Keyboard, Joystick). \\
    Mouse & Captures mouse input, representing the current state. \\
    Keyboard & Caputes keyboard input, representing key states. \\
    DeviceRelay & Relays recieved human input from server to the OS. \\
    MouseRelay & Relays mouse movement to the OS. \\
    KeyboardRelay & Relays pressed keys to the OS. \\
    \hline
  \end{tabular}
  \caption{Class purpose definition}
  \label{tab:classes}
\end{table}

\begin{figure}
  \centering
  \input uml/sd-system.tex
  \caption{Server and client sequence diagram}
\end{figure}

\begin{figure}
  \centering
  \input uml/sd-clipboard.tex
  \caption{Clipboard sharing sequence diagram}
\end{figure}
